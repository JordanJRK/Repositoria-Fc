\documentclass[a5paper, 11pt]{article}
\usepackage[utf8]{inputenc}
\usepackage{xcolor}
\usepackage[left=3cm,right=2.5cm]{geometry}
\usepackage{fancyhdr}
\usepackage{graphicx}
\definecolor{jordan}{RGB}{164, 157, 250}

\pagestyle{fancy}
    \fancyhf{}
    \rhead{\color{yellow}{Iron Fist}}
    \chead{}
    \lhead{\leftmark}
    %Para poner cosas a la derecha
    
    \rfoot{\today}
    \cfoot{\thepage}
    \lfoot{}
    


\begin{document}
\begin{center}
\textcolor{yellow}{\textbf{\Huge{IRON FIST}}}    
\end{center}


\pagecolor{jordan}

\color{white}


\section*{\underline{Información sobre la serie}}

\subsection*{\color{orange}{Camino del Iron Fist}}

La serie se desarrolla dentro del universo de superheroes de Marvel y la primera temporada (que es la que he visto) transcurre en un periodo de 15 años en la que el protagonista Dani Rand irá descubriendo que todo lo que conoce, inculo el ser Iron Fist, lo involucra a él y su familia más de lo que él pensaba.

La forma en la que se desarrolla la serie puede sentirse al desde el comienzo y hasta poco antes de la mitad como demasiado lenta, más teniendo en cuenta que los pocos momentos que realmente son relevantes se encuentran en momentos muy puntuales de episodios en los que se tocan los temas por muy poco tiempo. 
Sin embargo ésto no representa ningun inconveniente ni podría tomarse como un error, puesto que justo que se encuentren tan reducidos es lo que hace que durante la mayoría del tiempo no se te ocurra que lo que sucederá despues o, incluso algo que ya viste y no consideraste relevante, es de hecho un punto importante de la historia. 
Justo esta forma de contar la historia hace que la serie, aunque es centrada en los superheroes, sea entretenida no por la acción, ya que justo no es de los puntos fuertes de esta sino que es apenas suficiente como para mantener el interes, sino que lo realmente relevante y que te mantiene enganchado es enterarte de que forma irán a parar las consecuencias de las desiciones tomadas por el protagonista.

Un punto importante a tomar en cuenta sobre la serie es que tiene muchas partes que se centran en el drama y la historia en su mayoría tiene que ver con el mundo empresarial y las relaciones que se tienen estando en ese entorno, tocando temas como las consecuencias en el ámbito familiar, en la salud, en los tratos fuera de la ley y otras realidades que muchas veces no se nos vienen a la mente cuando pensamos en las personas que son los encargados de sacar adelante una compañía, por lo que, \underline{la serie es altamente recomendada si eres una persona}
\\ \underline{que disfruta de conocer a los personajes, sus relaciones}
\\ \underline{y las motivaciones de estos de una forma realista} y si \\ también no tienes inconveniente con invertir tu tiempo en episodios de alrededor de 45 minutos.

\subsection*{\color{red}{Detrás de Iron fist}}

Esta serie fue producida por Marvel y dirigida por Scott Buck, que también trabajó dirigiendo otras series, una para Marvel llamada ``Inhumans" y otra en el 2006 con el titulo de ``Dexter", siendo estas dos las más conocidas, emitida la primera temporada a partir del 17 de Marzo de 2017 y estando disponible a partir del 16 de Marzo de 2022 en Disney+.


El elenco (junto con los personajes) está formado principalmente por:
\begin{enumerate}
    \item Finn Jones/Juan Pablo Muñoz(Actor de doblaje) 
        \begin{itemize}
            \item[*] {\fcolorbox{green}{yellow}{\color{black}{IRON FIST}}} Heroe entrenado en artes marciales y control del chi con la responsabilida de proteger la entrada a un templo sagrado de K'un-Lun de cualquiera que represente un peligro.
            \item[*] {\fcolorbox{yellow}{black}{Danny Rand}} Huerfano que es dado por muerto después de un accidente de avión en el que iba su familia, una con gran poder económico y fundadora de una empresa con su apellido que se volvería importante con el paso del tiempo.
            \end{itemize}
    \item Jessica Henwick/Gigliola Mariangel(Actriz de doblaje)
         \begin{itemize}
            \item[*] {\fcolorbox{red}{black}{Colleen Wing}}
            Profesora de artes marciales que entrena a jóvenes de su comunidad para inducirlos a buscar un mejor futuro fuera de lo que les ofrece en las calles.
            \end{itemize}
    \item Tom Pelphrey/Eyal Meyer(Actor de doblaje)
        \begin{itemize}
            \item[*] {\fcolorbox{cyan}{black}{Ward Meachum}}
            Hijo del co fundador de la empresa Rand que junto con su hermana tiene que hacerse cargo de la empresa después de la muerte de su padre.
            \end{itemize}
    \item Jessica Stroup/María Doris Cuevas(Actriz de doblaje)
        \begin{itemize}
            \item[*] {\fcolorbox{pink}{black}{Joy Meachum}}
            Hija del co fundador y hermana de Word la cual apoya y anima a su hermano para continuar con la empresa que les quedó después de la muerte de su padre.
            \end{itemize}
    \item David Wenham/Ignacio Leyton(Actor de doblaje)
        \begin{itemize}
            \item[*] {\fcolorbox{orange}{black}{Harold Meachum}}
            {\underline{Amigo del padre de Danny}} y co fundador de la empresa Rand, padre de dos hijos que se entera que sufre de una enfermedad muy grave tiempo después de la muerte de su socio.
            \end{itemize}
    \item Rosario Dawson/Cecilia Valenzuela(Actriz de doblaje)
        \begin{itemize}
            \item[*] {\fcolorbox{white}{black}{Claire Temple}}
            Enfermera que es afectada por la delincuencia y que se ve involucrada con superheroes donde sea que vaya.
            \end{itemize}
    \item Carrie-Ann/Maureen Herman(Actriz de doblaje)
        \begin{itemize}
            \item[*] {\fcolorbox{blue}{black}{Jeryn "Jeri" Hogarth}}
            Vieja conocida de Danny y muy cercana a la familia Rand en el tiempo que aún estaban dirigiendo la empresa.
            \end{itemize}
    \item Sacha Dhawan/Javier Jiménez(Actor de doblaje)
        \begin{itemize}
            \item[*] {\fcolorbox{brown}{black}{Davos / Steel Serpent}}
            Compañero de Danny durante su entrenamiento en K'un-Lun y aspirante fallido para convertirse en Iron Fist.
            \end{itemize}
    \item Ramón Rodríguez/Felipe Waldhorn(Actor de doblaje)
        \begin{itemize}
            \item[*] {\fcolorbox{violet}{black}{Bakuto}}
            Maestro experto en artes marciales y {\underline{sensei de Colleen antes de conocer a Danny}} que tiene una extraña facinación con el origen, poder, historia y deber del Iron Fist
            \end{itemize}
    \item Wai Chin Ho/Rosario Zamora(Actriz de doblaje)
        \begin{itemize}
            \item[*] {\fcolorbox{green}{black}{Madame Gao}}
            Líder de un grupo muy peligroso de mafiosos que tiene el control de la ciudad de Nueva York, del cual solo tienen conocimiento las personas que llevan el tiempo suficiente como para saber que es mejor no intervenir en sus planes.
            \end{itemize}
            
\begin{figure}[h]
\begin{flushright}
    \caption{logo de iron fist}
    \includegraphics[scale=0.1,angle=15]{imagenes/81Yu8IcxvAL._AC_SL1500_.jpg}
    \end{flushright}
\end{figure}   
\end{enumerate}

\newpage

\section*{{\underline{\color{pink}{¿Tienes que verla? y porque si}}}}

\color{blue}{Es muy sencillo, no hay alguna serie que trate estos temas y que los convine tan bien, la imagen que da acerca de las cosas que podrían ocurrir en una familia que está a cargo de una empresa es una de las cosas más interesantes que he podido ver, y siendo sincero no se muestra de una forma exagerada, sino que todo lo contrario, se me hace muy verosimil la cantidad de estrés a la que están sometidos.} \color{red}{Otra muy buena razon es que se puede ver como Danny va dandose cuenta como han cambiado las cosas y uno realmente puede empatizar con él} \color{brown}{(o al menos yo lo hice)} \color{red}{en cuanto a la inosencia con la que vuelve después de tiempo y considera aún a esas personas con las que creció como todavía cercanas a él, como si el tiempo no hubiera pasado y no fueran personas totalmente distintas a las que fueron cuando eran niños.} \color{violet}{Y ya si se quiere una razón más en plan tema de super heroes, es una de las historias pilares para el grupo de heroes concido como Defenders, en el cual aparecen algunos otros como Dare Devil y Luke Cage, que en el caso de Dare Devil ya ha aparecido en otros proyectos de Marvel como la pelicula de Spiderman.}


\newpage

\begin{figure}[t]
    \centering
    \caption{Danny Rand}
    \includegraphics[scale=0.3,angle=18]{imagenes/f4e249c36b5a0230c119fd5ada2dd75dcab66bf8.jpg}
\end{figure}

\begin{figure}[h]
    \centering
    \caption{Madamme Gao}
    \includegraphics[scale=0.4,angle=-8]{imagenes/258005.jpg}
\end{figure}
\newpage

\end{document}
